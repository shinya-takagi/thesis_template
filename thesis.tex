% ------------------------------------------------
%     Preamble
% ------------------------------------------------
\usepackage{luatexja} % LuaTeX with Japanese
\usepackage{ifplatform} % Check your OS, Windows, MacOS, Linux, and Cygwin.
% In WSL, it detects your OS as Linux.
% Compile your tex file with option '-shell-escape' is required.

% Font
\usepackage[no-math]{fontspec}
\ifwindows% For Windows User
  \newcommand{\SetJapaneseFont}{yu-win10}
\fi
\ifmacosx% For MacOS User
  \newcommand{\SetJapaneseFont}{hiragino-pro}
\fi
\iflinux%
  \newcommand{\SetJapaneseFont}{haranoaji}
\fi
\usepackage[\SetJapaneseFont,no-math]{luatexja-preset}

% ----- Mathematics -----------------
\usepackage{amsmath,amsfonts}
\usepackage{bm} % Bold font in Math mode
\usepackage{chngcntr} % counter with section number
\counterwithin{equation}{section}
\counterwithin{figure}{section}
\counterwithin{table}{section}

% Physics and Chemistry
\usepackage{siunitx} % SI unit
\DeclareSIUnit{\barn}{\textrm{b}} % barn is deprecated in siunitx pacakge.
\usepackage{physics}
\usepackage[version=4]{mhchem}

% Graphic
\usepackage{graphicx} % required for using graphic
\usepackage{float} % use [H] option in figure environment.
\usepackage{subcaption} % small caption in main figure
\captionsetup[subfigure]{labelformat=simple}
\renewcommand{\thesubfigure}{(\alph{subfigure})} % refer to subcaption with parentheses.

% Adjusting spaces (With showframe option to show range of space)
\usepackage[top=20truemm, bottom=25truemm, left=25truemm, right=15truemm]{geometry}

% Reference
% Reference with BibLaTeX, Check: https://qiita.com/shiro_takeda/items/fac1351495f32c224a28
% \usepackage[backend=biber, style=numeric-comp]{biblatex}
\usepackage[backend=biber, style=numeric-comp]{biblatex-japanese} % For BibLaTeX with Japanese documents
\ExecuteBibliographyOptions{
  sorting=none,sortcites=true,
  abbreviate=true,date=year,
  isbn=false,url=false,
  doi=false,backref=true,
  maxnames=2,minnames=1
}
% \DeclareNameAlias{author}{given-family} % author name output GivenName FamilyName.
\addbibresource{ref_master.bib} % Your bibliography
\usepackage[setpagesize=false]{hyperref}
\usepackage{tabularx} % usual table for
\usepackage{multirow} % merge row
\usepackage{url}
\usepackage{xcolor} % Set color for hyperref package
\hypersetup{hidelinks, luatex}

\usepackage{cleveref}
% For cleveref package
\crefname{equation}{式}{式} % {環境名}{単数形}{複数形} \crefで引くときの表示
\crefname{figure}{図}{図}
\crefname{table}{表}{表}
\crefname{algorithm}{Algorithm}{Algorithm}
\crefname{section}{第}{第}
\creflabelformat{section}{#2#1節#3}
\crefname{subsection}{第}{第}
\creflabelformat{subsection}{#2#1小節#3}
\crefname{appendix}{付録}{付録}
\newcommand{\crefpairconjunction}{と} % A and B
\newcommand{\crefrangeconjunction}{から} % From A to B
\newcommand{\crefmiddleconjunction}{,} % A, B, and C
\newcommand{\creflastconjunction}{,および}

% Multi-line comment out, use comment environment.
\usepackage{comment}

% Insert space between Chapter and title
\makeatletter
\def\@makechapterhead#1{%
  \vspace*{2\Cvs}% English 50pt
  {\parindent\z@\raggedright\normalfont%
    \ifnum\c@secnumdepth>\m@ne%
      \huge\headfont\@chapapp\thechapter\@chappos\quad%
      %% \par\nobreak
      %% \vskip \Cvs % English 20pt
    \fi
    \interlinepenalty\@M%
    \Huge \headfont{} #1\par\nobreak%
    \vskip 3\Cvs}} % English 40pt
\makeatother

%----------- Original Command -------------------->
% Create new command: \newcommand
% Change the command: \renewcommand
\newcommand{\ThesisTitle}{論文のタイトル}
\newcommand{\ThesisTitleEnglish}{Thesis Title for English}
\newcommand{\ThesisType}{修士論文}
\newcommand{\ThesisYear}{2023}
\newcommand{\ThesisAuthor}{著者\quad{名}}

\newcommand{\Nuclide}[2]{$^{#1}#2$} % Put Nuclide such Fl-298(\Nuclide{298}{114}) for New superheavy element

% Technical Term